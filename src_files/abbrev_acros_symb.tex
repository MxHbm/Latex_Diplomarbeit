\glsenableentrycount % aktiviert \cgls, \cglspl, \cGls, \cGlspl, siehe https://tex.stackexchange.com/questions/98494/glossaries-dont-print-single-occurences/230664#230664

% Abkürzungen, die im Abkürzungsverzeichnis auftauchen und automatisch durch das glossaries-Paket sortiert werden
\newacronym[description={Container loading problem}]{CLP}{CLP}{container loading problem}
\newacronym[description={Vehicle routing problem}]{VRP}{VRP}{vehicle routing problem}
\newacronym[description={Capacitated vehicle routing problem}]{CVRP}{CVRP}{capacitated vehicle routing problem}
\newacronym[description={Bin packing problem}]{BPP}{BPP}{bin packing problem}
\newacronym[description={Last-in-first-out}]{LIFO}{LIFO}{last-in-first-out}
\newacronym[description={Cutting and packing}]{CaP}{C\&P}{cutting and packing}
\newacronym[description={Constraint programming}]{CP}{CP}{constraint programming}
\newacronym[description={Strip packing}]{SP}{SP}{strip packing}
\newacronym[description={Mixed-integer programming}]{MIP}{MIP}{mixed-integer programming}
\newacronym[description={Linear programming}]{LP}{LP}{linear programming}
\newacronym[description={Genetic algorithm}, plural = GAs, firstplural = genetic algorithms]{GA}{GA}{genetic algorithm}
\newacronym[description={Evolutionary Algorithm}]{EA}{EA}{evolutionary algorithm}
\newacronym[description={Tabu search}]{TS}{TS}{tabu search}
\newacronym[description={Local search}]{LS}{LS}{local search}
\newacronym[description={Simulated annealing}]{SA}{SA}{simulated annealing}
\newacronym[description={Greedy randomized adaptive search procedure}]{GRASP}{GRASP}{greedy randomized adaptive search procedure}
\newacronym[description={Adaptive Large Neighborhood Search}]{ALNS}{ALNS}{adaptive large neighborhood search}
\newacronym[description={Deepest-Bottom-Left-Fill}]{DBLF}{DBLF}{deepest-bottom-left-fill}
\newacronym[description={Bottom-Left-Fill}]{BLF}{BLF}{bottom-left-fill}
\newacronym[description={Maximum Touching Perimeter}]{MTP}{MTP}{maximum touching perimeter}
\newacronym[description={Large neighborhood search}]{LNS}{LNS}{large neighborhood search}
\newacronym[description={Pallet loading problem}]{PLP}{PLP}{pallet loading problem}
\newacronym[description={Machine learning}]{ML}{ML}{machine learning}
\newacronym[description={Artificial neural network}]{ANN}{ANN}{artificial neural network}
\newacronym[description={Feedforward neural network}]{FFNN}{FFNN}{feedforward neural network}
\newacronym[description={Reinforcement Learning}]{RL}{RL}{reinforcement learning}
\newacronym[description={Iterated Local Search}]{ILS}{ILS}{iterated local search}
%\newacronym[description={Support vector machine}]{SVM}{SVM}{support vector machine}
\newacronym[description={Logistic regression}]{LR}{LR}{logistic regression}
\newacronym[description={Two-dimensional loading capacitated vehicle routing problem}]{2L-CVRP}{2L--CVRP}{two-dimensional loading capacitated vehicle routing problem}
\newacronym[description={Three-dimensional loading capacitated vehicle routing problem}]{3L-CVRP}{3L--CVRP}{three-dimensional loading capacitated vehicle routing problem}
\newacronym[description={Three-dimensional loading vehicle routing problem with time windows}]{3L-VRPTW}{3L--VRPTW}{three-dimensional loading vehicle routing problem with time windows}
\newacronym[description={Time windows}]{TW}{TW}{time windows}
\newacronym[description={Lower bound}]{LB}{LB}{lower bound}
\newacronym[description={Upper bound}]{UB}{UB}{upper bound}
\newacronym[description={Manual last-in-first-out}]{MLIFO}{MLIFO}{manual last-in-first-out}
\newacronym[description={Load-bearing strength}]{LBS}{LBS}{load-bearing strength}


% Abkürzungen, die nicht im Abkürzungsverzeichnis aufgeführt werden
%\newabbreviation{zB}{z.\,B.}{zum Beispiel}

\glsaddall[types=abbreviation]

% Symbole die im Symbolverzeichniss erscheinen sollen
% Mit "F5" kompilieren oder "Tools -> Befehle -> makeglossaries" (F9) starten, um das Symbolverzeichnis zu aktualisieren
% \newsymb{<sort by>}{<name>}{<symbol>}{<unit>}
%
\begin{comment}
\newsymb{E}{Erwartungswert}{\mathbb{E}\left(\cdot\right)}{}
\newsymb{P}{Wahrscheinlichkeitsmaß}{\mathbb{P}\left(\cdot\right)}{}
\newsymb{V}{Varianz}{\mathbb{V}\left(\cdot\right)}{}
\newsymb{X}{Zufallsvariable}{X}{}
%
% Oder so verwenden und im Fließtext dann mit $\ExpValue$ arbeiten:
%\newcommand{\ExpValue}{\mathbb{E}\left(\cdot\right)}
%\newsymb{E}{Erwartungswert}{\ExpValue}{}
%
\glsaddall[types={symbols}] % Alle Symbole werden dem Symbolverzeichnis hinzugefügt
\end{comment}