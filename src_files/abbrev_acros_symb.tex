\glsenableentrycount % aktiviert \cgls, \cglspl, \cGls, \cGlspl, siehe https://tex.stackexchange.com/questions/98494/glossaries-dont-print-single-occurences/230664#230664

% Abkürzungen, die im Abkürzungsverzeichnis auftauchen und automatisch durch das glossaries-Paket sortiert werden
% ===============================
% Optimization & Problem Types
% ===============================
\newacronym[description={Container Loading Problem}]{CLP}{CLP}{container loading problem}
\newacronym[description={Pallet Loading Problem}]{PLP}{PLP}{pallet loading problem}
\newacronym[description={Bin Packing Problem}]{BPP}{BPP}{bin packing problem}
\newacronym[description={Strip Packing}]{SP}{SP}{strip packing}
\newacronym[description={Cutting and Packing}]{CaP}{C\&P}{cutting and packing}
\newacronym[description={Vehicle Routing Problem}]{VRP}{VRP}{vehicle routing problem}
\newacronym[description={Capacitated Vehicle Routing Problem}]{CVRP}{CVRP}{capacitated vehicle routing problem}
\newacronym[description={Two-dimensional Loading Capacitated Vehicle Routing Problem}]{2L-CVRP}{2L--CVRP}{two-dimensional loading capacitated vehicle routing problem}
\newacronym[description={Three-dimensional Loading Capacitated Vehicle Routing Problem}]{3L-CVRP}{3L--CVRP}{three-dimensional loading capacitated vehicle routing problem}
\newacronym[description={Three-dimensional Loading Vehicle Routing Problem with Time Windows}]{3L-VRPTW}{3L--VRPTW}{three-dimensional loading vehicle routing problem with time windows}
\newacronym[description={Time Windows}]{TW}{TW}{time windows}
\newacronym[description = {Relative Percentage Deviation}]{RPD}{RPD}{relative percentage deviation}

% ===============================
% Optimization Methods & Models
% ===============================
\newacronym[description={Linear Programming}]{LP}{LP}{linear programming}
\newacronym[description={Mixed-Integer Programming}]{MIP}{MIP}{mixed-integer programming}
\newacronym[description={Constraint Programming}]{CP}{CP}{constraint programming}
\newacronym[description={Lower Bound}]{LB}{LB}{lower bound}
\newacronym[description={Upper Bound}]{UB}{UB}{upper bound}

% ===============================
% Metaheuristics & Search Algorithms
% ===============================
\newacronym[description={Genetic Algorithm}, plural=GAs, firstplural=genetic algorithms]{GA}{GA}{genetic algorithm}
\newacronym[description={Evolutionary Algorithm}]{EA}{EA}{evolutionary algorithm}
\newacronym[description={Local Search}]{LS}{LS}{local search}
\newacronym[description={Iterated Local Search}]{ILS}{ILS}{iterated local search}
\newacronym[description={Tabu Search}]{TS}{TS}{tabu search}
\newacronym[description={Simulated Annealing}]{SA}{SA}{simulated annealing}
\newacronym[description={Greedy Randomized Adaptive Search Procedure}]{GRASP}{GRASP}{greedy randomized adaptive search procedure}
\newacronym[description={Large Neighborhood Search}]{LNS}{LNS}{large neighborhood search}
\newacronym[description={Adaptive Large Neighborhood Search}]{ALNS}{ALNS}{adaptive large neighborhood search}
\newacronym[description={Random Route Generation}]{RRG}{RRG}{random route generation}
\newacronym[description = {Best-Known Solution}]{BKS}{BKS}{best-known solution}

% ===============================
% Packing Heuristics
% ===============================
\newacronym[description={Deepest-Bottom-Left-Fill}]{DBLF}{DBLF}{deepest-bottom-left-fill}
\newacronym[description={Bottom-Left-Fill}]{BLF}{BLF}{bottom-left-fill}
\newacronym[description={Maximum Touching Perimeter}]{MTP}{MTP}{maximum touching perimeter}
\newacronym[description={Maximum Touching Area}]{MTA}{MTA}{maximum touching area}
\newacronym[description={Last-In-First-Out}]{LIFO}{LIFO}{last-in-first-out}
\newacronym[description={Manual Last-In-First-Out}]{MLIFO}{MLIFO}{manual last-in-first-out}
\newacronym[description={Load-Bearing Strength}]{LBS}{LBS}{load-bearing strength}
\newacronym[description={Loading Flag}, plural=LFLs]{LFL}{LFL}{loading flag}
\newacronym[description={Loading Status}]{LST}{LST}{loading status}

% ===============================
% Machine Learning & AI
% ===============================
\newacronym[description={Machine Learning}]{ML}{ML}{machine learning}
\newacronym[description={Artificial Neural Network}]{ANN}{ANN}{artificial neural network}
\newacronym[description={Feedforward Neural Network}]{FFNN}{FFNN}{feedforward neural network}
\newacronym[description={Reinforcement Learning}]{RL}{RL}{reinforcement learning}
\newacronym[description={Logistic Regression}]{LR}{LR}{logistic regression}
\newacronym[description ={eXtreme Gradient Boosting}]{XGB}{XGB}{extreme gradient boosting}
%\newacronym[description={Support vector machine}]{SVM}{SVM}{support vector machine}

% ===============================
% Evaluation Metrics (Confusion Matrix & ROC)
% ===============================
\newacronym[description={True Positive}]{TP}{TP}{true positive}
\newacronym[description={False Negative}]{FN}{FN}{false negative}
\newacronym[description={False Positive}]{FP}{FP}{false positive}
\newacronym[description={True Negative}]{TN}{TN}{true negative}
\newacronym[description={True Positive Rate}]{TPR}{TPR}{true positive rate}
\newacronym[description={False Positive Rate}]{FPR}{FPR}{false positive rate}
%\newacronym[description={Receiver Operating Curve}]{ROC}{ROC}{receiver operating curve}
\newacronym[description={Area Under the Receiver Operating Curve}]{AUROC}{AUROC}{area under the receiver operating curve}
\newacronym[description={Matthews Correlation Coefficient}]{MCC}{MCC}{matthews correlation coefficient}
\newacronym[description={Mutual Information}]{MI}{MI}{mutual information}
\newacronym[description={Fisher Score}]{F-Score}{F-Score}{fisher score}
\newacronym[description={Analysis of Variances}]{ANOVA}{ANOVA}{analysis of variances}
% Abkürzungen, die nicht im Abkürzungsverzeichnis aufgeführt werden
%\newabbreviation{zB}{z.\,B.}{zum Beispiel}

\glsaddall[types=abbreviation]

% Symbole die im Symbolverzeichniss erscheinen sollen
% Mit "F5" kompilieren oder "Tools -> Befehle -> makeglossaries" (F9) starten, um das Symbolverzeichnis zu aktualisieren
% \newsymb{<sort by>}{<name>}{<symbol>}{<unit>}
%
\begin{comment}
\newsymb{E}{Erwartungswert}{\mathbb{E}\left(\cdot\right)}{}
\newsymb{P}{Wahrscheinlichkeitsmaß}{\mathbb{P}\left(\cdot\right)}{}
\newsymb{V}{Varianz}{\mathbb{V}\left(\cdot\right)}{}
\newsymb{X}{Zufallsvariable}{X}{}
%
% Oder so verwenden und im Fließtext dann mit $\ExpValue$ arbeiten:
%\newcommand{\ExpValue}{\mathbb{E}\left(\cdot\right)}
%\newsymb{E}{Erwartungswert}{\ExpValue}{}
%
\glsaddall[types={symbols}] % Alle Symbole werden dem Symbolverzeichnis hinzugefügt
\end{comment}