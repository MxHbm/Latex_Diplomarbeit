\chapter{Mathematical formulation}
\label{chap:mathematical_formulation}

This two-index vehicle flow formulation is taken from \cite{tamke_branch-and-cut_2024}.\footcite[cf.][pp. 6-7]{tamke_branch-and-cut_2024}
The advantage is that infeasible tours of the subproblem \gls{CLP} are excluded
as route candidates for the masterproblem \gls{VRP}.

\subsection*{Sets}
A directed graph $G=(N,A)$ is defined  with $N$ is the set of nodes including
the depot and $C$ only the set of customers excluding the depot. The arc set is
defines as $A = \{ (i, j) \mid i, j \in N \}$. The Sets $\delta^+$ and $\delta^-$ define
the arcs, which are leaving respective entering a node. For example, for node 3, $\delta^+ = \{(3,5)\}$ and $\delta^-= \{(2,3)\}$.
The set $A(S)$ defines a sequence of arcs connecting nodes in a random set of customers $C$,
which is a route in wider sense. For example the set $A(S)=\{(2,1),(1,5),(5,3)\}$ connects nodes
2 and 3 via node 5. Every set $A(S)$ is defined by the total volume $q(S)$ and total weight $v(S)$
demanded by all customers $i \in S$, with $S \subseteq C$. Set $A(P)$ includes all infeasible sequences $P$ leading
to an infeasible route.

\begin{table}[ht]
    \centering
    \begin{tabular}{llc}
        \toprule
        Symbol        & Description                              & Range                         \\
        \midrule
        $N$           & Set of nodes                             & $\{ 0, \dots, N \}$           \\
        $C$           & Set of customers                         & $\{ 1, \dots, N \} $          \\
        $A$           & Set of directed arcs                     & $\{(0,1), \dots, (n-1, n) \}$ \\
        $\delta^+(i)$ & Set of arcs leaving node $i$             & $\in A$                       \\
        $\delta^-(i)$ & Set of arcs entering node $i$            & $\in A$                       \\
        $A(S)$        & Set of arcs with both ends in subset $S$ & $\in A$                       \\
        $A(P)$        & Set of arcs in an infeasible path $P$    & $\in A$                       \\
        \bottomrule
    \end{tabular}
\end{table}
\vspace{0.2em}

\subsection*{Parameters}
The parameters are the costs $c_{ij}$ for travelling from one node $i$ to another node $j$, and
are symmetrical defined, stating $c_{ij} \overset{!}{=} c_{ji}$. Additionally the number of vehicles
available for the fleet planning is given as constant $K$. No solution is allowed to have
more routes than avalaible vehicles. $r(S)$ defines a \gls{LB} for the numbers of vehicles needed
to include all customers in $S$. The \gls{LB} is calculated as the ceiling of the larger
value between the total weight $q(S)$ divided by the weight capacity $Q$, and the
total volume $v(S)$ divided by the volume capacity $V$:
\[r(S) = \max\left( \left\lceil \frac{q(S)}{Q} \right\rceil, \left\lceil \frac{v(S)}{V} \right\rceil \right)\]

\begin{table}[ht]
    \centering
    \begin{tabular}{ll}
        \toprule
        Symbol   & Description                                                   \\
        \midrule
        $c_{ij}$ & Cost or distance of traveling from node $i$ to node $j$       \\
        $K$      & Number of available vehicles                                  \\
        $Q$      & Weight limit of vehicle                                       \\
        $V$      & Volume limit of vehicle                                       \\
        $q(S)$   & Total weight requested by customers in $S$                    \\
        $v(S)$   & Total volume requested by customers in $S$                    \\
        $r(S)$   & Minimum number of vehicles required to cover customers in $S$ \\
        \bottomrule
    \end{tabular}
\end{table}

\bigskip

\subsection*{Decision Variable}
For the Two-Index-Flow model only one binary decision variable is necessary, which states, if the arc
between nodes $i$ and $j$ is included in the solution.
\[
    x_{ij} =
    \begin{cases}
        1 & \text{if arc } (i,j) \text{ is used in the solution} \\
        0 & \text{otherwise}
    \end{cases}
\]

\bigskip

\subsection*{Model}
\textbf{Objective:}
\begin{equation}
    \label{eq:objective}
    \min \sum_{(i,j)\in A} c_{ij}x_{ij}
\end{equation}

\textbf{Subject to:}
\begin{align}
    \sum_{(i,j)\in \delta^+(i)} x_{ij} & = 1                                                                                  &  & \forall i \in C                         \label{eq:flowout}     \\
    \sum_{(i,j)\in \delta^-(i)} x_{ij} & = 1                                                                                  &  & \forall i \in C                         \label{eq:flowin}      \\
    \sum_{(i,j)\in \delta^+(0)} x_{ij} & \leq K                                                       \label{eq:vehiclelimit}                                                                     \\
    \sum_{(i,j)\in A(S)} x_{ij}        & \leq |S| - r(S)                                                                      &  & \forall S \subseteq C, S \neq \emptyset \label{eq:capacitycut} \\
    \sum_{(i,j)\in A(P)} x_{ij}        & \leq |A(P)| - 1                                                                      &  & \forall \text{ infeasible paths } P     \label{eq:pathelim}
\end{align}

The objective function minimizes the total costs and the travelled distance incorporated
by each arc distance between nodes (\ref{eq:objective}). The constraints \ref{eq:flowin}
and \ref{eq:flowout} are restricting that every customer can only be visited once by defining the number
of arcs leaving and entering a node $i$ to 1. These two constraints do not apply to the depot,
which is included in every tour. Constraint~\ref{eq:vehiclelimit} limits the maximum number of
tours/ vehicles to $K$ by limiting the number of arcs entering the depot ($(i,j)\in \delta^+(0)$).
The next constraint~\ref{eq:capacitycut} states, that the number of arcs within the subset $A(S)$
must be smaller or equal to the number of customers minus the \gls{LB} for vehicles. This constraint
ensures, that routes are connected to the depot and that each subset $S$ is not violating
the volume and weight capacities. For example, if $S=\{1,2,3\}$ with $A(S)=\{(2,3),(3,1)\}$
and $r(S) = 1$, then $2 \leq 3 - 1$ holds, but if the \gls{LB} would be instead 2 then, $2 \not\leq 3 - 1$
and two tours need to be constructed. The last constraint~\ref{eq:pathelim} excludes all infeasible paths
$P$ from the solution, every path is defined by a subset of arcs $A(P)$ defining it. Infeasible
paths are labeled as such, when the corresponding \gls{CLP} to a route is not sufficient with loading
constraints. The constraint determines, that at least one binary variable defining an arc
from the set of arcs defining the infeasible path $|A(P)|$ has to have the value 0, that the left
side of this constraint is smaller equal to the right side.




