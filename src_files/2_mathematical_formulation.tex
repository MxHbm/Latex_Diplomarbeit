\section{Mathematical Formulation}
\label{sec:mathematical_formulation}

This two-index vehicle flow formulation for the \gls{3L-CVRP} is taken from \cite{tamke_branch-and-cut_2024}.\footcite[cf.][p. 6f]{tamke_branch-and-cut_2024}
The advantage of this approach is that infeasible tours in the subproblem (\gls{CLP}) are automatically excluded as route candidates
in the master problem (\gls{VRP}), while loading constraints are not explicitly defined in the model but are instead enforced through
rules that eliminate infeasible routes during the solution process. Additionally, different subproblems can be defined and the model is
not rigid. Therefore, the used sets and prameters are presented and afterwards the constraints with explainations.

\subsection*{Sets}
A directed graph $G=(N,A)$ is defined by the set of nodes $N$ including the depot and the set of arcs
$A = \{ (i, i') \mid i, i' \in N \}$ between all nodes. The set of customer nodes $C$ is defined without the depot.
The Sets $\delta^+$ and $\delta^-$ define the arcs, which are leaving, respective entering a node.
For example, for node 3, $\delta^+ = \{(3,5)\}$ and $\delta^-= \{(2,3)\}$.
The set $A(S)$ defines a sequence of arcs connecting nodes in a random set of customers $C$,
which is a route in wider sense. For example the set $A(S)=\{(2,1),(1,5),(5,3)\}$ connects nodes
2 and 3 via node 5. Every set $A(S)$ is defined by the total volume $q(S)$ and total weight $v(S)$
demanded by all customers $i \in S$, with $S \subseteq C$. The set $A(P)$ includes all infeasible sequences $P$ leading
to an infeasible route. A sequence has to have the minimum length of two. All sets are summarized in Table~\ref{tab:set_definitions}.

\begin{table}[ht]
	\centering
	\begin{tabular}{llc}
		\toprule
		Symbol        & Description                              & Range                         \\
		\midrule
		$N$           & Set of nodes                             & $\{ 0, \dots, N \}$           \\
		$C$           & Set of customers                         & $\{ 1, \dots, N \} $          \\
		$A$           & Set of directed arcs                     & $\{(0,1), \dots, (n-1, n) \}$ \\
		$\delta^+(i)$ & Set of arcs leaving node $i$             & $\in A$                       \\
		$\delta^-(i)$ & Set of arcs entering node $i$            & $\in A$                       \\
		$A(S)$        & Set of arcs with both ends in subset $S$ & $\in A$                       \\
		$A(P)$        & Set of arcs in an infeasible path $P$    & $\in A$                       \\
		\bottomrule
	\end{tabular}
	\caption{Definitions of sets used in the two-index vehicle flow model formulation.}
	\label{tab:set_definitions}
\end{table}
\clearpage
\subsection*{Parameters}
The parameters can be divided in two groups, static and set-dependent.
The costs $c_{ii'}$ for travelling from one node $i$ to another node $i'$
are symmetrically defined. Furthermore, the triangle inequality
$c_{ii'} \leq c_{ih} + c_{hi'}$ needs to be fulfilled, stating the travel distance must always be shorter travelling
directly than via any intermediate node $h$. Additionally, the number of vehicles
available for the fleet planning is given as constant $K$. No solution is allowed to have
more routes than avalaible vehicles. $r(S)$ defines a \gls{LB} for the numbers of vehicles needed
to include all customers in $S$. All parameters are summarized in Table~\ref{tab:parameter_definitions}.
The \gls{LB} is calculated as the ceiling of the maximum
value between the total weight $q(S)$ divided by the weight capacity $Q$, and the
total volume $v(S)$ divided by the volume capacity $V$:
\[r(S) = \max\left( \left\lceil \frac{q(S)}{Q} \right\rceil, \left\lceil \frac{v(S)}{V} \right\rceil \right)\]
\begin{table}[ht]
	\centering
	\begin{tabular}{ll}
		\toprule
		Symbol    & Description                                                   \\
		\midrule
		$c_{ii'}$ & Cost or distance of traveling from node $i$ to node $i'$      \\
		$C$       & Total costs of the solution                                   \\
		$K$       & Number of available vehicles                                  \\
		$Q$       & Weight limit of vehicle                                       \\
		$V$       & Volume limit of vehicle                                       \\
		$q(S)$    & Total weight requested by customers in $S$                    \\
		$v(S)$    & Total volume requested by customers in $S$                    \\
		$r(S)$    & Minimum number of vehicles required to cover customers in $S$ \\
		\bottomrule
	\end{tabular}
	\caption{Parameter definitions for the two-index vehicle flow model formulation.}
	\label{tab:parameter_definitions}
\end{table}

\subsection*{Decision Variable}
For the model formulation only one binary decision variable is necessary indicating, if the arc
between nodes $i$ and $i'$ is included in the solution.
\[
	x_{ii'} =
	\begin{cases}
		1 & \text{if arc } (i,i') \text{ is used in the solution} \\
		0 & \text{otherwise}
	\end{cases}
\]
\clearpage

\subsection*{Model}
\textbf{Objective:}
\begin{equation}
	\label{eq:objective}
	\min \, C = \sum_{(i,i')\in A} c_{ii'}\cdot x_{ii'}
\end{equation}

\textbf{Subject to:}
\begin{align}
	\sum_{(i,i')\in \delta^+(i)} x_{ii'} & = 1                                                                                  &  & \forall i \in C                         \label{eq:flowout}     \\
	\sum_{(i,i')\in \delta^-(i)} x_{ii'} & = 1                                                                                  &  & \forall i \in C                         \label{eq:flowin}      \\
	\sum_{(i,i')\in \delta^+(0)} x_{ii'} & \leq K                                                       \label{eq:vehiclelimit}                                                                     \\
	\sum_{(i,i')\in A(S)} x_{ii'}        & \leq |S| - r(S)                                                                      &  & \forall S \subseteq C, S \neq \emptyset \label{eq:capacitycut} \\
	\sum_{(i,i')\in A(P)} x_{ii'}        & \leq |A(P)| - 1                                                                      &  & \forall \text{ infeasible paths } P     \label{eq:pathelim}
\end{align}

The objective function minimizes the total costs and the travelled distance. The constraints \ref{eq:flowout} and \ref{eq:flowin}
are restricting that every customer can and must only be visited once by defining the number
of arcs leaving and entering a node $i$ to 1. These two constraints do not apply to the depot,
which is included in every tour. Constraint~\ref{eq:vehiclelimit} limits the maximum number of
tours to $K$ by limiting the number of arcs entering the depot ($(i,i')\in \delta^+(0)$).
The next constraint~\ref{eq:capacitycut} states, that the number of arcs within the subset $A(S)$
must be smaller or equal to the number of customers minus the \gls{LB} for vehicles. This constraint
ensures, that routes are connected to the depot and that each subset $S$ is not violating
the volume and weight capacities. For example, if $S=\{1,2,3\}$ with $A(S)=\{(2,3),(3,1)\}$
and $r(S) = 1$, then $2 \leq 3 - 1$ holds, but if the \gls{LB} would be instead 2 then, $2 \not\leq 3 - 2$
and two tours need to be constructed. The last constraint~\ref{eq:pathelim} excludes all infeasible paths
$P$ from the solution. Every path $P$ is defined by a subset of arcs $A(P)$ defining it. Infeasible
paths are labeled as such, when the corresponding \gls{CLP} of a route is infeasible due to the loading
constraints. The constraint determines, that at least one binary variable defining an arc
from the set of arcs defining the infeasible path $|A(P)|$ has to have the value 0, that the left
side of this constraint is smaller equal to the right side. The set of infeasible paths is determined by
solving the \gls{CLP}.




