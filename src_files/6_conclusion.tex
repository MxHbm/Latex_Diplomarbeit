\chapter{Conclusion and Outlook}
\label{chap:conclusion}
\begin{comment}
The variety of \gls{CLP} applications in logistics is wide, and the urge to consider practical constraints
and realistic datasets is high. For practitioners, heuristics are the first choice as good solutions
can be retrieved in satisfactory time. Exact solutions help to understand the structure of optimal solutions
and to evalutate and enhance existing heuristics. To accelerate exact methods different \gls{ML} enhancements
were presented and the \gls{3L-CVRP} was identified as a promising future use case for \gls{ML} classifiers.
The importance of realistic constraints of the \gls{CLP} were higlighted and the challenges faced when
training the classifier with such constraints were discussed. A dataset from \textcite{krebs_advanced_2021} was identified
was selected for future training of a classifier. To train a \gls{3L-CVRP} classifier that incorporates practical constraints, several next steps are
required. First, a training dataset must be created, consisting of individual tours that are pre-labeled
as feasible or infeasible using an exact \gls{CP} approach. In subsequent training iterations and epochs,
relevant features for accurately predicting feasibility must be evaluated. This includes assessing how
challenging it is to encode each individual constraint as a feature. Once the classifier achieves a
satisfactory level of accuracy, it can be integrated into a complete \gls{3L-CVRP} algorithm. The
classifier will then be used to predict the feasibility of solutions, enabling a comparison between
algorithmic performance with and without the classifier. Finally, the model can be tested on additional
problem instances—such as those introduced in Chapter 5—to evaluate its generalization capabilities.
This will allow the creation of meaningful benchmarks across multiple datasets.
% New start
The wide range of \gls{CLP} applications in logistics underscores the growing need to incorporate practical
constraints and realistic datasets. While heuristics remain the preferred choice for practitioners due to
their efficiency in producing good solutions within acceptable time frames, exact methods play a crucial
role in understanding the structure of optimal solutions and benchmarking heuristics.
To enhance exact solution approaches, two \gls{ML}-based examples have been presented, and the potential of
including classifiers in the solution process has been discussed. Main take aways are, that the training dataset
needs to be carefully selected including features, which can display the complex realistics \gls{CLP} constraints,
and representative data points, which are used also in the algorihtm the classifier is integrated into.
The \gls{3L-CVRP} has been identified as a promising candidate for the application of \gls{ML} classifiers,
as many tours need to be checked for packing feasibility and realistic constraints are needed to consider
practical logistical scenarios. The dataset from \textcite{krebs_advanced_2021} was
identified as a suitable foundation for future work training a classifier.
The next steps involve creating a labeled training dataset by creating \gls{VRP} tours, which are retrieved
by solving the \gls{3L-CVRP} with an exact algorithm such as Branch\&Cut and storing the created tours.
Afterwards, these tours are classified as feasible or infeasible using an \gls{CP} model. The features
need to be selected in the following phase following an iterative process to identify representation methods
for the underlying \gls{CLP} constraints. Different \gls{ML} models will be trained and the most suiting
model will be selected based on the accuracy of the feasibility prediction. Once a satisfactory level of
accuracy is achieved, the classifier is integrated into an exact \gls{3L-CVRP} algorithm to predict
the feasibility of single tours and several tests about the influence are conducted.
Finally, the complete algorithm will be tested
on other \gls{3L-CVRP} datasets, presented in Chapter~\ref{sec:dataset_selection}, to assess its generalizability.
This procedure has the potential to yield meaningful benchmarks across multiple datasets, providing valuable insights
into the classifier's performance and its impact on the solution process.
\end{comment}

The wide range of \gls{CLP} applications in logistics underscores the growing need to incorporate practical
constraints and realistic datasets. While heuristics remain the preferred choice for practitioners due to
their efficiency in producing good solutions within acceptable time frames, exact methods are essential for
understanding the structure of optimal solutions and benchmarking heuristic performance.
To enhance exact solution approaches, two \gls{ML}-based examples were presented, illustrating the potential
of integrating classifiers into the solution process. A key takeaway is that the training dataset must be
carefully designed, both in terms of feature selection that captures complex, realistic \gls{CLP} constraints,
and the selection of representative data points which are similar to tours generated by the exact algorithm
in which the classifier is integrated. The \gls{3L-CVRP} has been identified as a promising use case for \gls{ML} classifiers, as
it involves evaluating many tours for packing feasibility while accounting for real-world logistical
constraints. The dataset from \krebsADataSetText has been identified as a suitable foundation
for future classifier training. The next steps involve generating a labeled training dataset by solving
instances of the \gls{3L-CVRP} using an exact algorithm such as branch-and-cut with underlying column generation and storing the resulting tours.
These tours, which represent a \gls{CLP} instance with constraints, will then be classified as feasible or
infeasible using a \gls{CP} model. Feature selection will follow an iterative process aimed at effectively
capturing the underlying \gls{CLP} constraints. Various \gls{ML} models will be trained, and the model that
provides the highest feasibility prediction accuracy will be selected. Once satisfactory accuracy is achieved,
the classifier will be integrated into the exact \gls{3L-CVRP} algorithm, used to generate the training data,
to substitute the exact time-demanding feasibility checker.
Subsequent experiments will assess the classifier’s impact on the algorithm’s performance. Finally, the complete
algorithm will be tested on additional \gls{3L-CVRP} datasets presented in Chapter~\ref{sec:dataset_selection}
to evaluate its generalizability. This process has the potential to yield meaningful benchmarks across
multiple datasets and provide valuable insights into the effectiveness of classifier integration in exact \gls{3L-CVRP} algorithms.