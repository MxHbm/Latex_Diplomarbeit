\chapter{Binary Classifier}
\label{chap:classifier}

A practical application of the \gls{CLP} is the integration in the \gls{VRP}, where
a number of customers need to be served with a set of items by a fleet of vehicles that have
to start from a depot and return. The goal is to minimize the total distance driven
by the vehicles. When considering multidimensional items, the NP-hard problem itself,
increases in complexity, as every tour is representing a \gls{CLP} itself. \footcite[cf.][pp. 1--2]{tamke_branch-and-cut_2024}

The emphasis is placed primarily on \gls{ML}-based classifiers.
These are supervised \gls{ML} algorithms predicting the
value of a categorical or binary output column, called label, based on the
values of other columns, called features. Classifiers learn from a labeled dataset,
where the correct output values are known in advance, and then use this knowledge to
make predictions on new, unseen data. The accuracy can be evaluated afterwards by comparing
the predicted labels with the actual labels. \footcite[cf.][]{kotsiantis_supervised_2007}
An exemplary train dataset is shown in Table~\ref{tab:classifier_label_data}.

\begin{table}[ht]
    \small
    \centering
    \begin{tabular}{@{}cccccc@{}}
        \toprule
        \textbf{Instance} & \textbf{Feature 1} & \textbf{Feature 2} & \dots  & \textbf{Feature n} & \textbf{Label} \\
        \midrule
        1                 & xxx                & x                  & \dots  & xx                 & True           \\
        2                 & xxx                & x                  & \dots  & xx                 & False          \\
        3                 & xxx                & x                  & \dots  & xx                 & True           \\
        \vdots            & \vdots             & \vdots             & \vdots & \vdots             & \vdots         \\
        \bottomrule
    \end{tabular}
    \caption[Exemplary training dataset for a classifier with known labels.]{Exemplary training dataset for a classifier with known labels. \footnotemark}
    \footnotetext{Adapted table from \textcite[p. 249]{kotsiantis_supervised_2007}.}
    \label{tab:classifier_label_data}
\end{table}


A classifier can be implemented using various \gls{ML} models such as \gls{LR},
\gls{ANN}, support vector machine, or others. However, the most crucial aspect of any
\gls{ML} model is the selection of data, particularly the choice of features and
the size of the training set, since many models can be easily preselected from available
libraries and be compared performance wise. The model attempts to learn correlations between the provided features
and the corresponding labels. If the features are poorly chosen, the model may fail
to capture the underlying patterns in the data. Additionally, if the training set
is too small, the model might not generalize well, ultimately lacking the ability
to accurately predict unseen data. Furthermore, it needs to be noted, that available
\gls{ML} models are not by nature superior to other models, but can significantly outperform
other models on specific application problem \footcite[cf.][pp. 250, 264]{kotsiantis_supervised_2007}.


\subsubsection{Objectices for ML approaches}
To compare different models and \gls{ML} approaches identical objectives are needed. The most common
measures rely on the binary output of the confusion matrix, which sorts the output along
their true and predicted labels.

\begin{table}[ht]
    \centering
    \begin{tabular}{@{}lcc@{}}
        \toprule
                                 & \textbf{Predicted Positive} & \textbf{Predicted Negative} \\
        \midrule
        \textbf{Actual Positive} & True Positive (TP)          & False Negative (FN)         \\
        \textbf{Actual Negative} & False Positive (FP)         & True Negative (TN)          \\
        \bottomrule
    \end{tabular}
    \caption{Confusion matrix (binary classification).}
    \label{tab:confusion_matrix}
\end{table}
The output can be used to calculate the accuracy and F1 score of the classification, which
helps to interpret the outcomes and compare different model types.

\[
    \text{Accuracy}=\frac{TP+TN}{TP+TN+FP+FN}
    \qquad
    \text{F1}=\frac{2\,TP}{2\,TP+FP+FN}
\]


\section{Data Retrieval}
\label{sec:DataRetrieval}
Collecting the training data is one crucial step, for the model creation and two approaches exist.

\subsubsection{Saving labeled routes from exact algorithms}
The first one is to save all routes with exact loading label from an exisiting \gls{3L-CVRP}
algorithm, as \cite{zhang_learning-based_2022} did in his paper. However, in his approach it
had the advantage, that the binary model was supposed to be used in the same exact algorithm
and the solution structure of routes in the train data is similar to the real data afterwards. \footcite[cf.][]{zhang_learning-based_2022}
To follow this approach the exact branch-and-cut algorithm from \cite{tamke_branch-and-cut_2024} was
used to save all routes, which are labeled and found in a certain timelimit from an instance.
\footcites[Retrieved from][]{tamke_repository_2024}[cf.][]{tamke_branch-and-cut_2024} Every route
is sorted into a set composed of the loading flag and loading status. The flag represents which
subset of constraints was used for this tour and three flags were considered:
\begin{itemize}
    \item All Constraints $\mathcal{G}$
    \item No Support $\mathcal{G}\setminus \{\text{SupportArea}\}$
    \item No LIFO $\mathcal{G}\setminus \{\text{LIFO}\}$
\end{itemize}

The status defines the label for the loading retrived from the \gls{CP} Solver and can be either
Feasible, Infeasible, or Unknown. If the solver could not find a concrete Feasible/ Infeasible
label after a certain time limit the route will be classified as invalid. The following table
represents, which subsets of the loading flags and status are used for the train dataset.

\begin{table}[ht]
    \centering
    \begin{tabular}{@{}
            P{0.20\textwidth} % AllConstraints
            P{0.20\textwidth} % NoSupport
            P{0.20\textwidth} % NoLIFO
            @{}}
        \toprule
        \textbf{All Constraints}     & \textbf{NoSupport}           & \textbf{NoLIFO}              \\
        \midrule
        \cellcolor{green!20}Feasible & Feasible                     & Feasible                     \\
        \cellcolor{red!20}Infeasible & \cellcolor{red!20}Infeasible & \cellcolor{red!20}Infeasible \\
        \cellcolor{red!20}Unknown    & \cellcolor{red!20}Unknown    & \cellcolor{red!20}Unknown    \\
        \bottomrule
    \end{tabular}
    \label{tab:train_data_BC_routes}
    \caption{Constrution of training data from branch-and-cut routes. All green cells are interpreted as feasible, and all
        red cells as infeasible data}
\end{table}

As the trained classifier is used in an algorithm considering all loading constraints, feasible tours
must fulfill all loading constraints, but infeasible tours from a subset of loading constraints
will to a very high probability also be infeasible, when more loading constraints are considered.\footcite[cf.][p.7]{tamke_branch-and-cut_2024}
As the same tour can occur in different infeasible / unknown sets it is important to delete
duplicated tours with the same label before training the model.

\subsubsection{Creating random routes}
An alternative to the first approach, and is the only option when lacking a suiting complete algorithm, is to
create random routes from the \gls{3L-CVRP} instances used. As these random routes lack
a true label, if the corresponding loading of this route is feasible or infeasible, each random
route needs to be labeled individually. One of the approaches presented in Section~\ref{sec:classical_solution_approaches} can be chosen
for the \gls{CLP}, but the solution quality of heuristics must be high, or the best choice is to
use an exact approach to determine, if all items fit into the container with the loading constraints.
For this purpose, the exact \gls{CP} model was extracted from \cite{tamke_repository_2024} to create
a \gls{CP} solver, which can be called for just one route.\footcite[Stolen with permission from][]{tamke_repository_2024}
As there are no control mechanisms in comparison to the exact algorithm, the Algorithm~\ref{alg:rand_routes_generation} was created to test different parameter
combinations. As the pseudocode has a lot of details, the most important characteristics and
functionalities are summarized in the following:

\begin{itemize}
    \item 4 Input Parameters
          \begin{itemize}
              \item Multiplier $\alpha$: Integer determining repetitions ($n\times\alpha$) for each route length $n$
              \item  Attempts limit $\beta$: Integer determining max failures per repetition
              \item  Success threshold $\gamma$: Integer determining success threshold per repetition
              \item Instance set $\mathcal{I}$: Set of instances routes should be generated from
          \end{itemize}
    \item Route length is starting with two customers increasing until no routes can be found
    \item Every new random route is checked, if the route is already contained in a route set
    \item Algorithm stops when no random route could be drawn in $n \times \alpha$ rounds
\end{itemize}

Generally the higher the parameters $\alpha$, $\beta$ and $\gamma$ are set and the more instances
are considered in the instance set $\mathcal{I}$, the more routes will be found. Different
random training datasets are presented in the Chapter~\ref{chap:computational_study}.

\section{Model Training}
\label{sec:ModelTraining}

As the datasets are quite large, the mini-batch training approach was chosen for every model.

\section{Features}
\label{sec:Features}
