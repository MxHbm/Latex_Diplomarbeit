\chapter{Algorithm}
\label{chap:algorithm}
The focus for the algorithm was not to implement the best metaheuristic available for both routing and loading, but
to use a simple and adaptable algorithm. The focus is primarily on presenting approaches, how a trained classifier can
be integrated into exisiting \gls{3L-CVRP} algorithms and what are the impacts on the algorithm. Therefore the \Gls{ILS}
algorithm was chosen, a moderate and simple algorithm used regularly in the \gls{VRP} literature. First, the principal
algorithm and the neighborhoods will be explained. Second, the emphasis is set on the feasibility check for loading and
on the single elements of the algorithm, where a check is necessary. Third, different strategies for the usage of the
classifier are presented.

\section{Principal Algorithm}
\label{sec:algorithm}
The \gls{ILS} was proposed by \cite{lourenco_iterated_2003} and has the goal to incorporate simplicity and generality.
As the authors say, metaheuristics became recently too sophisticated and problem specific and lack the purpose of
being applied to different use cases. The main concept is based on iterations of the metaheuristic, in which the current
solution is first perturbated (diversification) and afterwards improved by different sequential local search neighborhoods
(intensification). The solution process restarts with the best solution after a number of iterations without improvement
or continues with the current worse solution to explore new areas of the solution space.
The proposed base algorithm is depicted in Algorithm~\ref{alg:base_ILS}. Here, $s^*$ stands for the best solution, $s\sp{\prime}$
for a perturbated neighbor of $s^c$ and $s\sp{c}$ for the current solution. Note that the symbol $s^c$
is used as a placeholder, and shows the current state of the solution after each step. So the current solution
$s^c$ before the perturbation is usually different from $s^c$ after the local search, ss non-deterministic
perturbation and local search can return to the same solution, ambiguity exists to a certain extent.
\begin{algorithm}[ht]
    \caption{General Iterated Local Search Algorithm}\label{alg:base_ILS}
    \begin{algorithmic}[1]
        \Procedure{ILS}{}
        \State $s_0 = \text{GenerateInitialSolution()}$
        \State $s^*,\,s^c = \text{LocalSearch}(s_0)$
        \While{$ \text{stop condition not met}$} \Comment{Total run time}
        \State $s'  = \text{Perturbation}(s^c, history)$
        \State $s\sp{c}  = \text{LocalSearch}(s\sp{\prime})$
        \State $s^c = \text{Acceptance Criterion}(s^*,\,s\sp{c}, history)$
        \EndWhile
        \State \textbf{return} $s^*$
        \EndProcedure
    \end{algorithmic}
\end{algorithm}
It must be noted, that the constructive heuristics, neighborhoods for perturbation and local search, need to be fitted for
the use case. The acceptance criterion can have several different implementations. To name a few, the algorithm could always
generate a random restart after a certain number of iterations without improvement, set back to the actual best solution
obtained or incorporating a \gls{SA} component accepting also more worse solutions during the runs. The major challenge is
to find a fitting perturbation mechanism, which allows to explore new solution space areas, without falling back to previous solutions
after the local search. Therefore it should be guaranteed, that several iterations can be used for diversification and
intensification before resetting the algorithm. History is refered in the algorithm to some knowledge base,
to save the values of previos solution to skip already reached solutions or the iterations number without improvement since the
last restart or solution improvement.\footcite[cf.][]{lourenco_iterated_2003}
\parbreak
The presented base \gls{ILS} was adapted for this work to obtain good solutions for the \gls{3L-CVRP}, where for the \gls{VRP}
the overall costs need to be minimized, and every single tour needs to represent a feasible packing of the \gls{CLP}.
Therefore the following algorithm~\ref{alg:principal_ILS} was constructed which has the same symbols for solutions. The perturbation
is of dual nature, perturbating the current solution with $K$ random moves of one neighborhood as standard, and after
several iterations ($a\times b$) without an improvement of the best solution, with $K$ random moves each from all neihghboods. In the \gls{LS}
the selected neighborhoods are sequentially walked through updating the current solution for each feasible improvement on
the current solution. The acceptance criterion controls the solution process updating the current solution, until a new overall
best solution was found or the limit for iterations without improvement $a$ is reached, then the current solution is resetted
to the best solution found. It needs to be investigated, if the random perturbation is able to leave this local optima,
when no improvements can be found or if this acceptance criterion is insufficient. But as the feasibility check of the loading
adds another CPU time demand, the algorithm should be kept simple.

\begin{algorithm}[ht]
    \caption{Iterated Local Search Algorithm}\label{alg:principal_ILS}
    \begin{algorithmic}[1]
        \Procedure{ILS}{$\text{AttemptsLimit}\,a,\, \text{RoundLimit}\,b$}
        \State $s_0 = \text{ModifiedSavings}()$ \Comment{After Zhang 2015}
        \State $s^*,\,s^c = \text{LocalSearch}(s_0)$
        \State $NoImpr \leftarrow 0$
        \While{$ \text{stop condition not met}$}
        \If{$NoImpr > a \times b$}
        \State $s\sp{\prime} = \text{BigPerturbation}(s^c)$ \Comment{Include all Perturbationneighborhoods}
        \Else
        \State $s\sp{\prime}  = \text{Perturbation}(s^c)$ \Comment{Include one Perturbationneighborhood}
        \EndIf
        \State $s\sp{c} = \text{LocalSearch}(s\sp{\prime})$
        \State{}
        \If{$s\sp{c}$ is better than $s^*$} \Comment{Acceptance Criterion}
        \State $s^* \gets s\sp{c}\,,\;NoImpr \gets 0$
        \Else
        \State $NoImpr \gets NoImpr + 1$
        \If{$NoImpr > a$}
        \State $s^c \gets s\sp{*}\,,\;NoImpr \gets 0$ \Comment{Reset current solution}
        \EndIf
        \EndIf
        \EndWhile
        \State \textbf{return} $s^*$
        \EndProcedure
    \end{algorithmic}
\end{algorithm}


The \gls{ILS} has a dual stop criterion, existing on one
time limit and a maximum number of iterations since the last improvement on the best solution was found. The second criterion
needed to be added to avoid exponential cycling of solutions, after the best solution was found.\footnote{Results compared with
    minimal costs due to \cite{tamke_branch-and-cut_2024}}. This results in four parameters to control the solution process:
\begin{itemize}
    \item Timelimit $t_{lim}$: Controls the total time of the algorithm including constructive and initial local search
    \item Maximum iterations without improvement $I_{max}$: Controls iterations number allowed to be cycling
    \item Attempts limit $a$: Controls how many iterations are proceeded, before resetting the current solution $s^c$
    \item Round limit $b$: Controls how many rounds (one round = $a$ iterations) are executed before big perturbation is applied
\end{itemize}

These parameters are then tuned in the Chapter~\ref{chap:computational_study} and different characteristics are already now
interpretable, if $a$ is large, then diversification is greater than the intensification, as the solutions is resetted less
times. If $b$ is small, then the diversification increases as the solutions are often perturbated more intensively, leading
to more potential improvements for the \gls{LS}. The two stop condition parameters have direct impact on the number of iterations
to be runned, but it is necessary to analyse the convergence of the solutions and adapt those parameters possibly to the
instance size.

\subsubsection{Constructive}
The initial solution is obtained by using the Savings Algorithm from \cite{clarke_scheduling_1964}. It is one of
most applied constructive algorithms for the \gls{VRP}, and is described by the following procedure. In the beginning
a single tour is created for every customer and the potential savings, when two customers are combined in one tour
($s_{ij} = d_{ij} - d_{0i} - d_{j0}$) are calculated.
Afterwards tours are merged until no negative savings are available, and the
procedure terminates with $K$ tours. \footcite[cf.][]{clarke_scheduling_1964} However, this constructive algorithm needs to be adapted
as the minimmum distance solution can exceed the maximum numbers of vehicles $K_{max}$ allowed per instance.
The modified savings approach was proposed by \cite{zhang_evolutionary_2015} to generate initial feasible solutions.
The modification is based on two steps, if the number of routes of the solution is exceeding the vehicle limit. First,
the merge procedure continue until $K = K_{max}$ or no feasible merges exist. Second, a repair procedure is invoked
by removing the routes with the least volume utilization from the solution and sorting the unassigned customers by
decreasing volume demand. Afterwards, the customers are tried to be reinserted at positions with the least
cost surplus. If no position could be found, the customer is inserted in a random chosen route by removing
other customers from this route before to create free capacity. This second procedure is repeated until the
number of routes $K = K_{max}$.\footcite[cf.][p.24]{zhang_evolutionary_2015}

\subsubsection{Neighborhoods}
\label{sec:neighborhoods}

The respected neighborhoods are divided in \textit{Intra}, changes on the customers within one tour, and \textit{Inter} neighborhoods,
changes between single tours. The following neighborhoods are implemented for each group:\footcite[cf.][pp. 89-90]{toth_vehicle_2014}

\begin{table}
    \centering
    \begin{tabular}{@{}P{0.3\textwidth}P{0.3\textwidth}@{}}
        \toprule
        \textbf{Intra} & \textbf{Inter} \\
        \midrule
        Swap           & Swap           \\
        Insertion      & Insertion      \\
        TwoOpt         &                \\
        \bottomrule
    \end{tabular}
\end{table}
Hereby, is a swap move defined by changing the position of two customers and insertion by removing one customer from one route
and replacing it in the same route (\textit{Intra}) or in another route (\textit{Inter}). A \textit{TwoOpt} move is characterized, when two arcs are deleted
from one route and are reinserted by swapping the indices of the arcs leading to a inversion of all customers between these two arcs.
For example, when the arcs $x_{12}$ are $x_{45}$ are selected the resulting arcs will be $x_{14}$ and $x_{25}$. To distinguish these
neighborhoods in the following section, they will be always called with their respective group, e.g. Interswap. Additionaly, a sixth
neighborhood is implemented deleting empty tours with no customers and is called \textit{DeleteEmptyRoutes}.

\parbreak

The neighborhoods for the perturbation are built upon the \textit{Inter} neighborhoods, and are characterized by $K$ random moves of
this respective neighborhood. Therefore they are called \textit{K\_RandomInterInsertions} and \textit{K\_RandomInterSwaps} in the
following. The perturbation neighborhood only consists of inter neighborhoods as the diversification of the solution is then
greater as with intra random moves. Regarding to \cite{lourenco_iterated_2003}, it is important, that perturbation holds the potential
to escape local optima and lays the foundation for a succesful \gls{LS}.\footcite[cf.][pp. 329f.]{lourenco_iterated_2003} The mechanism
how these two essential parts of the \gls{ILS} work will be demonstrated in the following section.

\section{Local Search and Perturbation}
\label{sec:LSandPerturbation}

In the \gls{LS} all selected neighborhoods are walked through sequentially. Every neighborhood is applied once and in the beginning
all moves, which lower the total costs of the solution are found. For the inter neighborhoods all routes are considered at the same
time and for intra neighborhoods each route is improved after each other.
Note, that loading is only respected with bound checks of the maximum volume and weight allowed per container at this step.
Afterwards all moves are sorted with decreasing savings for the route and is applied to the current solution. Now, it is tested
if the new constructed route aligns to the three dimensional loading constraints. This feasibility check will be discussed in
detail in the next section. If the new route is feasible, the \gls{LS} restarts for this solution resp. route, if it is infeasible
the move will be reverted and the next move will be applied. If there are no moves with negative savings left to apply, the search
terminates and continues with the next neighborhood or with the next route for the intra neighborhoods. After all neighborhoods
are walked through the resulting solution will be validated in the acceptance criterion. The procedure is visualized
in Figure~\ref{fig:LocalSearch} for one \gls{LS} neighborhood.
\begin{figure}[htbp]
    \centering
    {
        \begin{tikzpicture}[node distance=11mm and 16mm]
            % local defs for THIS picture only
            \begin{scope}
                % Nodes
                \node[dot] (start) {};
                \node[block, below=of start] (find) {Find all moves \\ for neighborhood $j$};
                \node[block, below=of find] (sort) {Sort moves due\\ to savings};
                \node[decision, below=of sort] (d1) {Iterate\\ over \\ Moves};
                \node[dot, left=30mm of d1] (leftdot) {};
                \node[block,  below=of d1] (undo)   {Undo\\ Move};
                \node[block,  right=of d1] (apply){Apply Move};
                \node[decisionY, below=of apply] (d2){Loading\\ Feasible?};

                % Arrows
                \draw[line] (start) -- (find);
                \draw[line] (find) -- (sort);
                \draw[line] (sort) -- (d1);
                \draw[line] (d1.west) -- node[pos=.45,above,text width=3cm,align=center]{\textbf{Termination}: No Moves left} (leftdot);
                \draw[line] (d1) -- (apply);
                \draw[line] (apply) -- (d2.north);
                \draw[line] (d2.west) -- node[pos =.45,above,align=center]{\textbf{False}} (undo.east);
                \draw[line] (undo.north) -- (d1.south);
                \coordinate (bump) at ($(d2.east)+(12mm,0)$); % how far to go right
                \draw[line]
                (d2.east) -- node[pos=0.5,above,align=center]{\textbf{True}}(bump)
                |- (find.east);

            \end{scope}
        \end{tikzpicture}
    }
    \caption{Local Search Procedure}
    \label{fig:LocalSearch}
\end{figure}


The perturbation buils upon a very similar approach. But in each neighborhood a certain number random moves $K$ must be found
before the perturbation terminates. The random move creation respects all moves which can be found between two different routes
and do not violate the maximum volume and weight limit of the vehicle. As shown in the \gls{ILS} Algorithm~\ref{alg:principal_ILS}
it is distinguished between a big and a normal perturbation. In the normal perturbation only the \textit{K\_RandomInterInsertions}
neighborhood is applied, and in the big additionally the \textit{K\_RandomInterSwaps} neighborhood. Other perturbation mechanism
like FourOpt or BlockInsertion were not considered as the Feasibility check of the loading is the limiting factor even though
the perturbation has bigger impact in creating new structural solutions.\footcite[cf.][pp. 329-332]{lourenco_iterated_2003}
Instead the presented algorithm holds two perturbation strategies with different strength.
The procedure is again visualized in Figure~\ref{fig:Perturbation}.


\begin{figure}[htbp]
    \centering
    {
        \begin{tikzpicture}[scale=0.65, transform shape,node distance=11mm and 16mm]
            % local defs for THIS picture only
            \begin{scope}
                % Nodes
                \node[dot] (start) {};
                \node[decision, below=of start] (d1) {Found\\ Moves \\ $< R$};
                \node[dot, left=20mm of d1] (leftdot) {};
                \node[block,  below=of d1] (gen)   {Generate\\ random Move};
                \node[block,  right=of gen] (apply){Apply Move};
                \node[decisionY, below=12mm of apply] (d2){Loading\\ Feasible?};
                \node[block,  below=of gen, left=18mm of d2] (undo) {Undo Move};

                % Arrows
                \draw[line] (start) -- (d1);
                \draw[line] (d1.west) -- node[pos=.45,above, align=center]{\textbf{True}} (leftdot);
                \draw[line] (d1.south) -- node[pos=.45,right,align=center]{\textbf{False}} (gen.north);
                \draw[line] (gen) -- (apply);
                \draw[line] (apply) -- (d2.north);
                \draw[line] (d2.west) -- node[pos =.45,above,align=center]{\textbf{False}} (undo.east);
                \draw[line] (undo.north) -- (gen.south);
                \coordinate (bump) at ($(d2.east)+(12mm,0)$); % how far to go right
                \draw[line]
                (d2.east) -- node[pos=0.5,above,align=center]{\textbf{True}}(bump)
                |- node[pos=.75,above,align=center]{\textbf{Found Moves ++}}
                (d1.east);

            \end{scope}
        \end{tikzpicture}
    }
    \caption{Perturbation procedure.}
    \label{fig:Perturbation}
\end{figure}


\section{Loading Feasibility Check}
\label{sec:FeasibilityCheck}
So far the description of the algorithm was reduced in minimizing the costs of the \gls{VRP} masterproblem and in vague
requirements, that new solutions or moves found need to be feasible regarding the three-dimensional loading constraints.
This check is time-consuming as every item need to placed accordingly in the container, testing all possible combinations.
The \gls{CLP} is $NP$-hard and as shown in \ref{sec:classical_solution_approaches}, most heuristics rely on fast, but not
optimal loading heuristics. The aim of the theses is to determine, if the usage of a binary classifier can bring cost and
speed advantages in combination with an exact \gls{CP} solver, in comparison to the baseline of only using the exact solver.
As this work is potential study, the combination with an exact solver is sufficient to derive insights for other use cases
leveraging either exact \gls{3L-CVRP} algorithms or exisiting metaheuristics with a classifier.
Before explaining how both tools can be integrated in the algorithm it is important to examine strengths and weaknesses of
each in the following overview:
\begin{figure}[ht]
    \centering
    \begin{minipage}[centering]{0.45\textwidth}
        \textbf{CP Solver}
        \begin{itemize}
            \item Returns exact loading status
            \item Computational heavy
            \item No need to verify solution
        \end{itemize}
    \end{minipage}
    \begin{minipage}[centering]{0.45\textwidth}
        \textbf{Binary Classifier}
        \begin{itemize}
            \item Fast to classify route
            \item Solution needs to be verified
            \item Adaptable acceptance threshold $y$
        \end{itemize}
    \end{minipage}
    \caption{Comparison between CP solver and binary classifier}
    \label{fig:comparison_classifier_cpSolver}
\end{figure}

The classifier has the potential to speed up the \gls{3L-CVRP} solution process, but accepted solutions need to be
verified with the \gls{CP} solver in a certain pattern to avoid false positives solutions during the algorithm.
Therefore the following four strategies are introduced, which determine how the loading feasibility can be checked and are
visualized in Figure~\ref{fig:tikz_four_variants}
\begin{figure}[!ht]
    \centering
    % First row
    \begin{minipage}[t]{0.45\textwidth}
        \centering
        \begin{tikzpicture}[node distance=11mm and 16mm,scale=0.75, transform shape]
            \begin{scope}
                % Nodes
                \node[dot] (start) {};
                \node[decisionY, below=of start] (d1) {Classifier\\Check};
                \node[block,  below=of d1] (Reject){Reject Tour};
                \node[dot, below=of Reject] (end) {};
                % Arrows
                \draw[line] (start) -- (d1);
                \draw[line] (d1) -- node[pos=0.5,left]{\textbf{False}}(Reject);
                \draw[line] (Reject) -- (end);
                \coordinate (bump) at ($(d1.east)+(10mm,0)$); % how far to go right
                \draw[line]
                (d1.east) -- node[pos=0.5,above]{\textbf{True}}(bump)
                |- (end.east);
            \end{scope}
        \end{tikzpicture}
        \caption{SpeedUp check.}
        \label{alg:tikz_variant1}
    \end{minipage}
    \hfill
    \begin{minipage}[t]{0.45\textwidth}
        \centering
        \begin{tikzpicture}[node distance=7mm and 16mm,scale=0.75, transform shape]
            \begin{scope}
                % Nodes
                \node[dot] (start) {};
                \node[decisionY, below=of start] (d1) {Classifier\\Check};
                \node[decisionCP, below=of d1] (d2) {CP\\Check};
                \node[block,  left=of d2] (Reject){Reject Tour};
                \node[dot, below=of d2] (end) {};
                % Arrows
                \draw[line] (start) -- (d1);
                \draw[line] (start) -- (d1);
                \draw[line] (d1) -- node[pos=0.5,left]{\textbf{True}}(d2);
                \draw[line] (d1) -| node[pos=0.25,above]{\textbf{False}}(Reject);
                \draw[line] (d2) -- node[pos=0.5,above]{\textbf{False}}(Reject);
                \draw[line] (d2) -- node[pos=0.5,right]{\textbf{True}}(end);
                \draw[line] (Reject) |- (end);
                \coordinate (bump) at ($(d1.east)+(10mm,0)$); % how far to go right
            \end{scope}
        \end{tikzpicture}
        \caption{Filter check.}
        \label{alg:tikz_variant2}
    \end{minipage}

    \vspace{2em} % space between rows

    % Second row
    \begin{minipage}[t]{0.45\textwidth}
        \centering
        \begin{tikzpicture}[node distance=11mm and 0mm,scale=0.75, transform shape]
            \begin{scope}
                % Nodes
                \node[dot] (start) {};
                \node[decision, below=of start] (d1) {SpeedUp\\Neighbor-\\hood?};
                \coordinate (mid_up) at ($(d1.south)$);
                \node[block] at ($(mid_up)+(-24mm,-8mm)$) (Filter){Use Filter \\ Variant};
                \node[block] at ($(mid_up)+(24mm,-8mm)$)(SpeedUp){Use SpeedUp \\ Variant};
                % midpoint between the bottoms of Filter and SpeedUp
                \coordinate (mid) at ($(Filter.south)!0.5!(SpeedUp.south)$);

                % place the end dot some distance below that midpoint
                \node[dot] (end) at ($(mid)+(0,-11mm)$) {};
                % Arrows
                \draw[line] (start) -- (d1);
                \draw[line] (d1) -| node[pos=0.35,above]{\textbf{False}}(Filter);
                \draw[line] (d1) -| node[pos=0.35,above]{\textbf{True}}(SpeedUp);
                \draw[line] (Filter.south) |- (end.west);
                \draw[line] (SpeedUp.south) |- (end.east);
            \end{scope}
        \end{tikzpicture}
        \caption{Hybrid check.}
        \label{alg:tikz_variant3}
    \end{minipage}
    \hfill
    \begin{minipage}[t]{0.45\textwidth}
        \centering
        \begin{tikzpicture}[node distance=11mm and 16mm,scale=0.75, transform shape]
            \begin{scope}
                % Nodes
                \node[dot] (start) {};
                \node[decisionCP, below=of start] (d1) {CP\\Check};
                \node[block,  below=of d1] (Reject){Reject Tour};
                \node[dot, below=of Reject] (end) {};
                % Arrows
                \draw[line] (start) -- (d1);
                \draw[line] (d1) -- node[pos=0.5,left]{\textbf{False}}(Reject);
                \draw[line] (Reject) -- (end);
                \coordinate (bump) at ($(d1.east)+(10mm,0)$); % how far to go right
                \draw[line]
                (d1.east) -- node[pos=0.5,above]{\textbf{True}}(bump)
                |- (end.east);
            \end{scope}
        \end{tikzpicture}
        \caption{NoClassifier check.}
        \label{alg:tikz_variant4}
    \end{minipage}

    \caption{Different feasibility check variants of ILS algorithm.}
    \label{fig:tikz_four_variants}
\end{figure}

The first approach is the \textit{SpeedUp} strategy, where only the classifier is used for checks during perturbation and
\gls{LS}. After $\omega$ iterations the current solution is checked with the exact solver and is either rejected or accepted.
The amount of false positive labeled routes is here crucial, as the probability to reject the solution grows likewise. It needs
to be studied, how the parameter $\omega$ influences this procedure as the chance exist, that infeasible tours become feasible
again during several iterations.
The second strategy is called \textit{Filter}, here classifier and \gls{CP} solver are called iteratively for every loading
feasibiliy check, the goal is to filter out infeasible routes before checking them exactly to save time in the exact check.
All false negative labeled routes represent a potential solution quality loss, as those tours will never be considered to be accepted.
The third strategy is a hybrid form between the last two presented, where the chosen strategy, \textit{Filter} or \textit{SpeedUp},
is dependent on the neighborhood structure (Perturbation, Intra- and Interneighborhood). The last and fourth strategy is the benchmark
for the other strategies and here no classifier is used during the procedure. The goal is to beat this strategy with well tuned
variations of the first ones. These variations are further shown in the Figure~\ref{fig:four_variants}. Here the $\bigstar$ symbol imply
the usage of the classifier and the $\clubsuit$ symbol to the \gls{CP} solver. The symbol combination mean the following:
\begin{itemize}
    \item $\clubsuit$: \gls{CP} Solver
    \item $\bigstar$: SpeedUp (Solely Binary Classifier)
    \item $\bigstar\clubsuit$: Filter
    \item \(\clubsuit \backslash \bigstar\clubsuit\): \gls{CP} Solver or Filter
    \item \(\bigstar\backslash\clubsuit\): \gls{CP} Solver or SpeedUp dependent on neighborhood
\end{itemize}

%Symbol macro: two stars
\newcommand{\twosym}{\ \(\bigstar\clubsuit\)}
%Symbol macro: two stars
\newcommand{\classsifiersym}{\ \(\bigstar\)}
%Symbol macro: two stars
\newcommand{\cpsym}{\ \(\clubsuit\)}

% Helper macro for algorithmic lines with symbols
\newcommand{\BothState}[1]{\State #1\twosym}
\newcommand{\CPState}[1]{\State #1\cpsym}
\newcommand{\ClassifierState}[1]{\State #1\classsifiersym}


\begin{figure}[h]
    \centering

    % First row
    \begin{minipage}[t]{0.45\textwidth}
        \textbf{Variant 1: SpeedUp}\par\vspace{0.5ex}
        \begin{algorithmic}[1]
            \State {ModifiedSavings \(\clubsuit \backslash \bigstar\clubsuit\)}
            \ClassifierState{InitialLocalSearch}
            \While{not stop condition}
            \ClassifierState{Perturbation}
            \ClassifierState{IntraLocalSearch}
            \ClassifierState{InterLocalSearch}
            \State {CP Check}
            \EndWhile
        \end{algorithmic}
        \captionsetup{type=algorithm,font=small}
        \captionof{algorithm}{SpeedUp Variant}
        \label{alg:variant1}
    \end{minipage}
    \hfill
    \begin{minipage}[t]{0.45\textwidth}
        \textbf{Variant 2: Filter}\par\vspace{0.5ex}
        \begin{algorithmic}[1]
            \BothState{ModifiedSavings}
            \BothState{InitialLocalSearch}
            \While{not stop condition}
            \BothState{Perturbation}
            \BothState{IntraLocalSearch}
            \BothState{InterLocalSearch}
            \EndWhile
        \end{algorithmic}
        \captionsetup{type=algorithm,font=small}
        \captionof{algorithm}{Filter Variant}
        \label{alg:variant2}
    \end{minipage}

    \vspace{2em} % space between rows

    % Second row
    \begin{minipage}[t]{0.45\textwidth}
        \textbf{Variant 3: Hybrid}\par\vspace{0.5ex}
        \begin{algorithmic}[1]
            \State {ModifiedSavings \(\clubsuit \backslash \bigstar\clubsuit\)}
            \State {InitialLocalSearch \(\bigstar\backslash \clubsuit\)}
            \While{not stop condition}
            \ClassifierState{Perturbation}
            \ClassifierState{IntraLocalSearch}
            \CPState{InterLocalSearch}
            \State {CP Check}
            \EndWhile
        \end{algorithmic}
        \captionsetup{type=algorithm,font=small}
        \captionof{algorithm}{Hybrid Variant}
        \label{alg:variant3}
    \end{minipage}
    \hfill
    \begin{minipage}[t]{0.45\textwidth}
        \textbf{Variant 4: NoClassifier}\par\vspace{0.5ex}
        \begin{algorithmic}[1]
            \CPState{ModifiedSavings}
            \CPState{InitialLocalSearch}
            \While{not stop condition}
            \CPState{Perturbation}
            \CPState{InterLocalSearch}
            \CPState{IntraLocalSearch}
            \EndWhile
        \end{algorithmic}
        \captionsetup{type=algorithm,font=small}
        \captionof{algorithm}{No Classifier Variant}
        \label{alg:variant4}
    \end{minipage}

    \caption{Four variants to implement classifier in the loading feasibility check ($\bigstar$
        does imply usage of classifier and $\clubsuit$ of the CP Solver).}
    \label{fig:four_variants}
\end{figure}

During the construction the \textit{SpeedUp} strategy can hardly be applied, which has two reasons. Firstly, this strategy relies
on rejecting solutions feasible with the classifier by the \gls{CP} Solver and resetting the solution to previous feasible solutions.
Secondly, the procedure of the modified savings algorihm is deterministic and repetitions will lead to the same solutions. In these
cases it needs to be investigated, if the \textit{Filter} strategy can speed up the solution process in the constructive without
great solution quality losses.

\parbreak

In the next chapter each strategy a computational study will be conducted for each strategy to compare them. To facilitate this approach,
a comparison of various published \gls{3L-CVRP} datasets will be conducted to compare and identify the most appropriate
dataset for training a binary feasibility classifier.

\subsection*{Parking Lot}
As discussed in Chapter~\ref{sec:classical_solution_approaches}, the verification of
packing feasibility for each individual route is computationally expensive.
Here, classifiers can significantly boost performance of existing exact algorithms by rapidly predicting the feasibility of the route. The
exact packing solution is then only computed for the final solution candidates or before an infeasible classified solution
is discarded to avoid incorrect eliminations, as presented above.
