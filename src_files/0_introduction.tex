\chapter{Introduction}
\label{sec:introduction}
In today’s globalized economy, efficient logistics networks are vital for managing the vast number of
goods transported daily. Maritime shipping, responsible for about $80\%$ of global trade,
forms a backbone of this network. \footcite[cf.][]{un_trade_and_development_unctad_review_2024}
However, the last mile, delivering individual parcels to customers, poses some of the most complex optimization
challenges. In 2022, over 161 billion parcels were delivered globally, equating to more than 5,000
deliveries every second.\footcite[cf.][]{statista_global_2022}
These numbers are only manageable through a strong logistics network, where even small inefficiencies
become very costly and optimization potential is huge. Two major challenges are faced in this domain.
First, the \gls{VRP}, which is the optimization of the routes driven by the delivery trucks and second,
the \gls{CLP}, which is the efficient arrangement of items of various shapes and sizes.
Several subproblems of the \gls{VRP} incorporate the dimensionality of the \gls{CLP}. The most basic form is the Capacitated VRP (CVRP),
which accounts for one-dimensional weight and volume constraints to prevent vehicle overloading.
More advanced formulations extend this to two- or three-dimensional packing considerations,
leading to the \gls{2L-CVRP} and \gls{3L-CVRP}, whose complexity increases exponentially with each added dimension.
When the \gls{VRP} and the \gls{CLP} are solved integrated,
solutions are on average $7\%$ more cost-efficient, but it is complex to solve
both problems as they are $NP$-hard. \footcite[cf.][p. 23]{cote_value_2016} Therefore, efficient heuristics are the practical
choice to find good solutions in a reasonable time, because exact solution methods are limited to
small instances considering only few practical constaints, as orientation rules, stability, and load-bearing limits. \footcite[cf.][p. 377f]{bischoff_issues_1995}
Recent research has shown, that \gls{ML} techniques can be used to accelerate the exact solution process
of the \gls{2L-CVRP}.\footcite[cf.][]{zhang_learning-based_2022}
This paper aims to provide an overview of the \gls{CLP} and explores how \gls{ML} can be used
to enhance the solution process of \gls{CLP} algorithms focusing on the \gls{3L-CVRP}.
The paper is structured as follows:
Chapter~\ref{chap:literature_review} introduces a general mathematical model for
the \gls{VRP} with CLP and its constraints as a subproblem as well as the as well as classical
solution approaches. Promising \gls{ML} approaches are discussed and a complete overview of algorithms
for \gls{3L-CVRP} algorithms are conducted. Then, Chapter~\ref{chap:classifier} introduces the required
steps for developing a binary classifier, ranging from data to modeling and the used features
are presented. Chapter~\ref{chap:algorithm} explains the implemented \gls{3L-CVRP} algorithm and
the different variants of it leveraging the usage of the binary classifier. The computational results and the
tuning of the parameters are shown in Chapter~\ref{chap:computational_study}. Additionally, Chapter~\ref{chap:application_krebs}
generalizes the usage of the presented algorithm and binary classifier by solving instances from a second dataset.
In Chapter~\ref{chap:critical_discussion} the used methods are critically evaluated and considerations for improvements
are discussed. This thesis ends with the conclusion
of the obtained results and outlook for further research in Chapter~\ref{chap:conclusion}.
