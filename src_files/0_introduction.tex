\chapter{Introduction}
\label{sec:introduction}
In today’s globalized economy, efficient logistics networks are vital for managing the vast number of
goods transported daily. Maritime shipping, responsible for about $80\%$ of global trade,
forms a backbone of this network. \footcite[cf.][]{un_trade_and_development_unctad_review_2024}
However, the last mile, delivering individual parcels to customers, poses some of the most complex optimization
challenges. In 2022, over 161 billion parcels were delivered globally, equating to more than 5,000
deliveries every second.\footcite[cf.][]{statista_global_2022}
These numbers are only manageable through a strong logistics network, where even small inefficiencies
become very costly and optimization potential is huge. Two major challenges are faced in this domain, which
is the optimization of the routes driven by the delivery trucks, the \gls{VRP}, and the efficient arrangement of items
of various shapes and sizes, the \gls{CLP}. When the \gls{VRP} and the \gls{CLP} are solved integrated,
solutions are on average around $7\%$ better cost-wise, but are complex to solve as
both problems are $NP$-hard. \footcite[cf.][p. 23]{cote_value_2016} Therefore, efficient heuristics are the practical
choice to find good solutions in a reasonable time, because exact solution methods are limited to
small instances considering only few practical constaints, as orientation rules, stability, and load-bearing limits. \footcite[cf.][pp. 377--378]{bischoff_issues_1995}
Recent research has shown that \gls{ML} techniques can be used to accelerate the exact solution process
of the \gls{2L-CVRP}.\footcite[cf.][]{zhang_learning-based_2022}
Therefore this paper aims to provide an overview of the \gls{CLP} and explore how \gls{ML} can be used
to enhance the solution process of \gls{CLP} algorithms focusing on the \gls{3L-CVRP}.
The paper is structured as follows:
Chapter~\ref{chap:literature_review} introduces a general mathematical model for
the \gls{VRP} with subproblem, the \gls{CLP} and its common constraints, as well as classical
solution approaches. Promising \gls{ML} approaches will be discussed and a complete overview of algorithms
for \gls{3L-CVRP} algorithms will be conducted. The Chapter~\ref{chap:classifier} introduces the required
steps for developing a binary classifier are presented, ranging from data to modeling and the used features
are presented. Chapter~\ref{chap:algorithm} explains the implemented \gls{3L-CVRP} algorithm leveraging
the usage of the binary classifier and classical solution approaches. The computational results and the
tuning of the used parameters are shown in \ref{chap:computational_study} and this thesis ends with the conclusion
of the obtained results and the outlook fur further research steps in Chapter~\ref{chap:conclusion}.
