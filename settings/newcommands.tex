%############## Own commands #####################´
\newcommand{\parbreak}{\vspace{\baselineskip}\noindent}
\newcommand{\gendreauDataSet}{Gendreau (2006)}
\newcommand{\mouraDataSet}{Moura (2009)}
\newcommand{\ceschiaDataSet}{Ceschia (2013)}
\newcommand{\krebsADataSet}{Krebs (2021a)}
\newcommand{\krebsBDataSet}{Krebs (2021b)}
\newcommand{\gendreauDataSetText}{\gendreauDataSet\ }
\newcommand{\mouraDataSetText}{\mouraDataSet\ }
\newcommand{\ceschiaDataSetText}{\ceschiaDataSet\ }
\newcommand{\krebsADataSetText}{\krebsADataSet\ }
\newcommand{\krebsBDataSetText}{\krebsBDataSet\ }


%Symbol macro: two stars
\newcommand{\twosym}{\ \(\bigstar\clubsuit\)}
%Symbol macro: two stars
\newcommand{\classsifiersym}{\ \(\bigstar\)}
%Symbol macro: two stars
\newcommand{\cpsym}{\ \(\clubsuit\)}

% Helper macro for algorithmic lines with symbols
\newcommand{\BothState}[1]{\State #1\twosym}
\newcommand{\CPState}[1]{\State #1\cpsym}
\newcommand{\ClassifierState}[1]{\State #1\classsifiersym}

\newcolumntype{P}[1]{>{\centering\arraybackslash}m{#1}} % left-aligned p{} helper

% --- Convenience wrappers (put after your tikzset) ---
%\newcommand{\House}[3]{\pic {house={#1/#2/#3}};}                 % outlined house
%\newcommand{\Person}[3]{\pic {person={#1/#2/#3}};}               % outlined person
%\newcommand{\HouseFilled}[3]{\pic {house_filled={#1/#2/#3}};}    % filled house
%\newcommand{\PersonFilled}[3]{\pic {person_filled={#1/#2/#3}};}  % filled person