%############## Own commands #####################´
\newcommand{\parbreak}{\vspace{\baselineskip}\noindent}
\newcommand{\gendreauDataSet}{Gendreau (2006)}
\newcommand{\mouraDataSet}{Moura (2009)}
\newcommand{\ceschiaDataSet}{Ceschia (2013)}
\newcommand{\krebsADataSet}{Krebs (2021a)}
\newcommand{\krebsBDataSet}{Krebs (2021b)}
\newcommand{\gendreauDataSetText}{\gendreauDataSet\ }
\newcommand{\mouraDataSetText}{\mouraDataSet\ }
\newcommand{\ceschiaDataSetText}{\ceschiaDataSet\ }
\newcommand{\krebsADataSetText}{\krebsADataSet\ }
\newcommand{\krebsBDataSetText}{\krebsBDataSet\ }

%Adds small bullet in front of text
\newcommand{\sbt}{\,\begin{picture}(-1,1)(-1,-3)\circle*{3}\end{picture} \ }

% Positioning something relative to someting else
\newcommand{\datasetPos}[2]{%
{#1} : {#2}
}

%Symbol macro: two stars
\newcommand{\twosym}{\ \(\bigstar\clubsuit\)}
%Symbol macro: two stars
\newcommand{\classsifiersym}{\ \(\bigstar\)}
%Symbol macro: two stars
\newcommand{\cpsym}{\ \(\clubsuit\)}

%Displaying hyperparameters better
\newcommand{\kv}[2]{\texttt{#1 = #2}}

% Helper macro for algorithmic lines with symbols
\newcommand{\BothState}[1]{\State #1\twosym}
\newcommand{\CPState}[1]{\State #1\cpsym}
\newcommand{\ClassifierState}[1]{\State #1\classsifiersym}

\newcolumntype{P}[1]{>{\centering\arraybackslash}m{#1}} % left-aligned p{} helper

% --- Convenience wrappers (put after your tikzset) ---
%\newcommand{\House}[3]{\pic {house={#1/#2/#3}};}                 % outlined house
%\newcommand{\Person}[3]{\pic {person={#1/#2/#3}};}               % outlined person
%\newcommand{\HouseFilled}[3]{\pic {house_filled={#1/#2/#3}};}    % filled house
%\newcommand{\PersonFilled}[3]{\pic {person_filled={#1/#2/#3}};}  % filled person


% ---------- Global style knobs ----------
\definecolor{numC}{RGB}{22,163,74} % green  (numerator)
\definecolor{denC}{RGB}{239,68,68} % red    (denominator)

% sizes (kept identical across all panels)
\newcommand{\ItemW}{0.8}   % item width
\newcommand{\ItemH}{0.6}   % item height
\newcommand{\ItemD}{0.8}   % item "depth/length"
\newcommand{\DistItemCont}{1.3}

\newcommand{\ContW}{2.3}   % container width
\newcommand{\ContH}{1.4}   % container height
\newcommand{\ContD}{1.4}   % container "depth/length"

% perspective offsets per unit depth (tweak once for the whole figure)
\newcommand{\dx}{0.52} % x-offset per unit depth
\newcommand{\dy}{0.35} % y-offset per unit depth

% ---------- Core drawing helpers ----------
% Draw a 2.5D cuboid from (x0,y0), with width w, height h, depth d.
% Fills/toplines are subtle; edges get a crisp outline.
\newcommand{\Cuboid}[7][]{% opts, x0, y0, w, h, d
    \begin{scope}[#1]
        \coordinate (A) at (#2,#3);
        \coordinate (B) at ($(A)+(#4,0)$);
        \coordinate (C) at ($(A)+(0,#5)$);
        \coordinate (D) at ($(A)+(#4,#5)$);
        \coordinate (S) at (\dx*#6,\dy*#6); % depth shift

        % Back rectangle (top) — light fill
        \fill[#7!3] ($(A)+(S)$) rectangle ($(D)+(S)$);

        % Front rectangle — a bit darker
        \fill[#7!6] (A) rectangle (D);

        % Connectors (sides)
        \fill[#7!10] (A) -- ($(A)+(S)$) -- ($(B)+(S)$) -- (B) -- cycle;       % bottom side
        \fill[#7!10] (B) -- ($(B)+(S)$) -- ($(D)+(S)$) -- (D) -- cycle;       % right side
        \fill[#7!10] (C) -- ($(C)+(S)$) -- ($(D)+(S)$) -- (D) -- cycle;       % top side

        % Crisp outlines (visible edges)
        \draw[black, line width=0.4pt] (A) rectangle (D);
        \draw[black, line width=0.4pt] (A) -- ($(A)+(S)$);
        \draw[black, line width=0.4pt] (B) -- ($(B)+(S)$);
        \draw[black, line width=0.4pt] (C) -- ($(C)+(S)$);
        \draw[black, line width=0.4pt] (D) -- ($(D)+(S)$);
        \draw[black, line width=0.4pt] ($(A)+(S)$) rectangle ($(D)+(S)$);
    \end{scope}
}

% Edge markers on one cuboid (placed on front-most edges).
%   width  = bottom-front edge
%   height = left-front vertical edge
%   length = right-leaning depth edge on top-front-right
\newcommand{\MarkWidth}[4]{% x0,y0,w,color
    \draw[line width=2.2pt, #4, rounded corners=0.6pt]
    (#1,#2) -- ++(#3,0);
}
\newcommand{\MarkHeight}[4]{% x0,y0,h,color
    \draw[line width=2.2pt, #4, rounded corners=0.6pt]
    (#1,#2) -- ++(0,#3);
}
% Mark the depth/length edge from top-right-front corner
\newcommand{\MarkLength}[4]{% x0,y0,d,color
    \coordinate (TRF) at (#1,0);%#2+\ItemH); % top-right-front
    \draw[line width=2.2pt, #4, rounded corners=0.6pt]
    (TRF) -- ++(\dx*#3,\dy*#3);
}

% Convenience wrappers that place item & container with consistent spacing
% They also expose anchors (I0,C0) for subsequent marks.
\newcommand{\PlaceItem}[2]{% x0,y0
    \coordinate (I0) at (#1,#2);
    \Cuboid{#1}{#2}{\ItemW}{\ItemH}{\ItemD}{black}
}
\newcommand{\PlaceContainer}[2]{% x0,y0
    \coordinate (C0) at (#1,#2);
    \Cuboid{#1}{#2}{\ContW}{\ContH}{\ContD}{black}
}

% ---------- Panel recipes ----------
% 1) Ratios between ITEM dimensions
\newcommand{\PanelItemDimDim}[3]{% cap, markA (W/H/L), markB (W/H/L), label
    \begin{subfigure}[t]{0.22\linewidth}
        \centering
        \begin{tikzpicture}[>=latex]
            \PlaceItem{0}{0}
            % choose axes for numerator/denominator
            % Numerator = green, Denominator = red
            \ifnum#2=1 \MarkWidth{0}{0}{\ItemW}{numC}\fi
            \ifnum#2=2 \MarkHeight{\ItemW}{0}{\ItemH}{numC}\fi
            \ifnum#2=3 \MarkLength{\ItemW}{0}{\ItemD}{numC}\fi

            \ifnum#3=1 \MarkWidth{0}{0}{\ItemW}{denC}\fi
            \ifnum#3=2 \MarkHeight{\ItemW}{0}{\ItemH}{denC}\fi
            \ifnum#3=3 \MarkLength{\ItemW}{0}{\ItemD}{denC}\fi
        \end{tikzpicture}
        \caption{#1}
    \end{subfigure}
}

% 2) Item vs Container dimension
\newcommand{\PanelItemVsCont}[2]{% caption, which axis: 1=W,2=H,3=L, label
    \begin{subfigure}[t]{0.32\linewidth}
        \centering
        \begin{tikzpicture}[>=latex]
            % place container slightly to the right
            \PlaceContainer{\DistItemCont}{0}
            \PlaceItem{0}{0}
            % numerator (item) green
            \ifnum#2=1 \MarkWidth{0}{0}{\ItemW}{numC}\fi
            \ifnum#2=2 \MarkHeight{\ItemW}{0}{\ItemH}{numC}\fi
            \ifnum#2=3 \MarkLength{\ItemW}{0}{\ItemD}{numC}\fi
            % denominator (container) red — marks drawn on its front edges
            \ifnum#2=1 \MarkWidth{\DistItemCont}{0}{\ContW}{denC}\fi
            \ifnum#2=2 \MarkHeight{\DistItemCont+\ContW}{0}{\ContH}{denC}\fi
            \ifnum#2=3 \MarkLength{\DistItemCont +\ContW}{0}{\ContD}{denC}\fi
        \end{tikzpicture}
        \caption{#1}
    \end{subfigure}
}

% 3) Height / Area  (area = item footprint W×L on front face surrogate)
\newcommand{\PanelHeightOverArea}[1]{%caption
    \begin{subfigure}[t]{0.22\linewidth}
        \centering
        \begin{tikzpicture}
            \PlaceItem{0}{0}
            % numerator: height (green)
            \MarkHeight{0}{0}{\ItemH}{numC}
            % denominator: area (red) -> outline the front face footprint
            \fill[denC!25]
            (0,0) --
            (\ItemW,0) --
            (\ItemW+\dx*\ItemD, \dy*\ItemD) --
            (\dx*\ItemD,\dy*\ItemD) -- cycle;
            \draw[denC, line width=0.5pt]
            (0,0) --
            (\ItemW,0) --
            (\ItemW+\dx*\ItemD, \dy*\ItemD) --
            (\dx*\ItemD,\dy*\ItemD) -- cycle;
            % thin red frame on the front face
            \draw[black, line width=0.4pt] (\ItemW,0) -- (\ItemW,\ItemH);
            % thin red frame on the front face
        \end{tikzpicture}
        \caption{#1}
    \end{subfigure}
}

% 4) Area / Container Area
\newcommand{\PanelAreaOverContArea}[1]{%caption
    \begin{subfigure}[t]{0.35\linewidth}
        \centering
        \begin{tikzpicture}
            \PlaceContainer{\DistItemCont}{0}
            \PlaceItem{0}{0}
            % numerator: item area (front face fill greenish)
            \fill[numC!25]
            (0,0) --
            (\ItemW,0) --
            (\ItemW+\dx*\ItemD, \dy*\ItemD) --
            (\dx*\ItemD,\dy*\ItemD) -- cycle;
            \draw[numC, line width=0.5pt]
            (0,0) --
            (\ItemW,0) --
            (\ItemW+\dx*\ItemD, \dy*\ItemD) --
            (\dx*\ItemD,\dy*\ItemD) -- cycle;
            \draw[black, line width=0.4pt] (\ItemW,0) -- (\ItemW,\ItemH);
            %Container  
            \fill[denC!25]
            (\DistItemCont,0) --
            (\ContW +\DistItemCont,0) --
            (\ContW+\DistItemCont+\dx*\ContD, \dy*\ContD) --
            (\dx*\ContD + \DistItemCont,\dy*\ContD) -- cycle;
            \draw[denC, line width=0.5pt]
            (\DistItemCont,0) -- (\ContW +\DistItemCont,0) -- (\ContW+\DistItemCont+\dx*\ContD, \dy*\ContD) -- (\dx*\ContD + \DistItemCont,\dy*\ContD) -- cycle;
            \draw[black, line width=0.4pt] (\ContW+\DistItemCont,0) -- (\ContW+\DistItemCont,\ItemH);
        \end{tikzpicture}
        \caption{#1}
    \end{subfigure}
}

% 5) Volume / Container Volume
\newcommand{\PanelVolumeOverContVolume}[1]{%caption
    \begin{subfigure}[t]{0.35\linewidth}
        \centering
        \begin{tikzpicture}
            \Cuboid{0}{0}{\ItemW}{\ItemH}{\ItemD}{numC}
            \Cuboid{\DistItemCont}{0}{\ContW}{\ContH}{\ContD}{denC}
        \end{tikzpicture}
        \caption{#1}
    \end{subfigure}
}