
\begin{figure}[ht]
    \centering
    \begin{tikzpicture}[
            scale = 0.6,
            transform shape,
            level distance=20mm,
            sibling distance=40mm,
            node/.style={rectangle, draw, align=center, rounded corners, minimum width=20mm, minimum height=10mm},
            selectedNode/.style={rectangle, draw,fill = myPetrol!30, align=center, rounded corners, minimum width=20mm, minimum height=10mm},
            edge from parent path={(\tikzparentnode.south) --($(\tikzparentnode.south)-(0,5mm)$) -| (\tikzchildnode.north)}
        ]

        \node[selectedNode] {Feature Selection}
        child {node[node](ch1) {Unsupervised \\ Learning}
            }
        child {node[selectedNode](ch2) {Supervised \\ Learning}
                child {node[selectedNode] (c4) {Filter Methods}
                        child {node[node] (c7) {Y:Numerical Data}}
                        child {node[node] (c8) {Y:Ordinal Data}}
                        child {node[selectedNode] (c9) {Y:Categorical Data}
                                child {node[selectedNode] (c10) {X:Numerical Data}
                                        child {node[selectedNode] (c13) {Fisher-Score}}
                                        child {node[selectedNode] (c14) {Mutual Information}}}
                                child {node[node] (c11) {X:Ordinal Data}}
                                child {node[node] (c12) {X:Categorical Data}}
                            }
                    }
                child {node[node] (c5) {Wrapper Methods}}
                child {node[node] (c6) {Embedded Methods}}
            };
    \end{tikzpicture}
    \caption{Feature selection with emphasis on filter methods.}
    \label{fig:overview_Feature_selection}
\end{figure}