
\begin{figure}[htbp]
    \centering
    {
        \begin{tikzpicture}[node distance=11mm and 16mm]
            % local defs for THIS picture only
            \begin{scope}
                % Nodes
                \node[dot] (start) {};
                \node[decision, below=of start] (d1) {Found\\ Moves \\ $< K$};
                \node[dot, left=20mm of d1] (leftdot) {};
                \node[block,  below=of d1] (gen)   {Generate\\ random Move};
                \node[block,  right=of gen] (apply){Apply Move};
                \node[decisionY, below=12mm of apply] (d2){Loading\\ Feasible?};
                \node[block,  below=of gen, left=18mm of d2] (undo) {Undo Move};

                % Arrows
                \draw[line] (start) -- (d1);
                \draw[line] (d1.west) -- node[pos=.5,above]{\textbf{True}} (leftdot);
                \draw[line] (d1.south) -- node[pos=.5,right]{\textbf{False}} (gen.north);
                \draw[line] (gen) -- (apply);
                \draw[line] (apply) -- (d2.north);
                \draw[line] (d2.west) -- node[pos =.5,above]{\textbf{False}} (undo.east);
                \draw[line] (undo.north) -- (gen.south);
                \coordinate (bump) at ($(d2.east)+(12mm,0)$); % how far to go right
                \draw[line]
                (d2.east) -- node[pos=0.5,above]{\textbf{True}}(bump)
                |- node[pos=.75,above]{\textbf{Found Moves ++}}
                (d1.east);

            \end{scope}
        \end{tikzpicture}
    }
    \caption{Perturbation Procedure}
    \label{fig:Perturbation}
\end{figure}
