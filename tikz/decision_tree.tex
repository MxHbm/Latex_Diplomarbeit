\begin{figure}[ht]
    \centering
    \begin{tikzpicture}[
            >=Latex,
            node distance=15mm and 0mm,
            decision/.style={rectangle, rounded corners=2pt, draw, fill=blue!8, align=center, inner sep=3pt},
            leafgood/.style={rectangle, rounded corners=2pt, draw, fill=green!10, align=center, inner sep=3pt},
            leafbad/.style={rectangle, rounded corners=2pt, draw, fill=red!10, align=center, inner sep=3pt},
            edge/.style={-Latex, thick}
        ]
        % Nodes
        \node[decision] (root) {Relative volume\\$> 70\%$?};

        \node[decision, below right=of root] (items) {Item count\\$> 30$?};
        \node[decision, below left=of root] (weight){Relative weight\\$> 80\%$?};

        \node[leafgood, below=of items, xshift=-14mm] (itemNo)  {Feasible};
        \node[leafbad, below=of items, xshift= 14mm] (itemYes) {Infeasible};

        \node[leafbad,  below=of weight, xshift=14mm] (wYes)   {Infeasible};
        \node[leafgood, below=of weight, xshift=-14mm] (wNo)    {Feasible};

        % Edges with labels
        \draw[edge] (root) -- node[above, sloped]{Yes} (items);
        \draw[edge] (root) -- node[above, sloped]{No}  (weight);

        \draw[edge] (items) -- node[left]{No}  (itemNo);
        \draw[edge] (items) -- node[right]{Yes} (itemYes);

        \draw[edge] (weight) -- node[right]{Yes} (wYes);
        \draw[edge] (weight) -- node[left]{No}  (wNo);

    \end{tikzpicture}
    \caption{Exemplary decision tree to understand classification.}
    \label{fig:decision_tree}
\end{figure}
